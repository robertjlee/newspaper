\documentclass[a4paper,DIV=11]{scrartcl}
\usepackage[british]{babel}
\usepackage{indentfirst}
\usepackage{multicol}
\usepackage{multirow}
\usepackage{textcomp}
\author{Robert Lee}
\renewcommand{\today}{March 2024}
\title{Newspaper Layout}
\subtitle{Reference Manual}

% Definition of term
%\newcommand{\deft}[2]{\nopagebreak\noindent\hspace{-2em}\textbf{#1}\par\noindent{}#2\pagebreak[1]\par}
\newcommand{\deft}[2]{\nopagebreak\noindent\hspace{0.5in}\textbf{#1}\par\noindent{}#2\pagebreak[1]\par}

% Properties key definition
\newcommand{\property}[5]{
  \subsection{#1: \texttt{#2}}
  \begin{itemize}
  \item Type: #3
  \item Default: #4
  \end{itemize}
  #5
}


\newcommand{\setting}{\property{Setting}}
\newcommand{\header}{\property{Header}}


% For CMR: DocBucket's workaround according to StackExchange:
\usepackage{textcomp}
\newcommand{\texttilde}{\raisebox{0.5ex}{\texttildelow}}



\begin{document}

\maketitle

\setlength{\parskip}{0.5ex plus 0.5ex}


\section{Abstract}

This manual discusses the reference implementation for a newspaper
typesetting algorithm.

The editor produces a number of inserts, including articles, titles,
pictures and so on, and they are typeset into an attractive newspaper
layout. This frees the editor from most typesetting tasks, while still
allowing them nearly full control over the process.


\section{Installation Requirements}

\deft{OpenJDK-11.0.2}{To use the programme, a Java Runtime
  Environment~(JRE) is required. A full Java Development Kit~(JDK) can
  be used to compile the source code. The programme is developed
  against Java~11; newer versions should also work. At time of
  writing, there are no external Java dependencies required at
  runtime.}

\deft{\LaTeXe}{Either PdfLaTex or LuaLaTex is recommended, and
  LuaLaTeX is strongly recommended if you want to use markdown in your
  input articles. This is used both to measure the size of articles
  prior to layout, and to perform the final typesetting of the document.
  The following packages are required:
  \begin{itemize}
  \item[multicol] Used by headspan articles
  \item[fontsize] Used to set up the default font size commands and associated spacing.
  \item[inputenc] Used to read UTF-8 characters in inputs
  \item[newtx] Supplies the default \textit{Times Roman} font
  \item[geometry] Used when measuring article sizes, and to set the
    final paper size.
  \item[babel] Used per default to set hyphenation points
  \item[indentfirst] Used per default to indent the first paragraph of
    each article.
  \item[markdown] Required only if using markdown input modes.
  \end{itemize}
  TODO:
  package version requirements.}

% \usepackage or \RequirePackage [opts]{name}[min y/m/d]

\subsubsection*{Optional Requirements}

\deft{JUnit-5.8.1 (for testing)}{The Java programme comes with a
  reasonably complete set of unit tests written under the JUnit
  framework. The framework is not required to compile or run the
  programme, but is fairly essential to further develop the
  application.}

\subsubsection*{The Programme}

The programme itself can be downloaded from:-

\texttt{https://github.com/robertjlee/newspaper}

\section{Article Requirements}

The aim of this package is to collect discrete inputs and to present them
in a professional-standard newspaper layout. Each input must have a
single file that the programme understands, although it may input
other files arbitrarily.

All inputs must be provided in \texttt{UTF-8} text encoding, one per
file, with special \textit{headers} provided at the start of the file,
as described later in this manual.\pagebreak[1]

The programme currently understands the following formats:-

\deft{\LaTeX}{The input will be typeset one or more times to
measure its length, and then once in the final output. While any
\LaTeX\ package (other than class files) can be used, some
require running the document multiple times to obtain the final
output. These are unlikely to work well, unless the overall
\textit{length} of the article is unchanged.}

\deft{Plain Text}{Documents in plain text are supported. You should
use a blank line to separate paragraphs; any
special spacing in the input will be lost. Most other characters
behave as you'd expect, although you can add extra dashes~(-) in the input
to produce longer dashes in the output (\texttt{1-{}-3 dashes -{}-{}-
  like this, or-else}; 1--3 dashes --- like this, or-else). You can
also use backticks~(\textasciigrave) and apostrophies~(\textasciiacute) to produce quotation marks
(\texttt{\textasciigrave\textasciigrave like this\textasciiacute\textasciiacute} for ``like this''). }

\deft{Markdown}{There is some support for markdown, which requires the
\texttt{lualatex} executable to be used, with the
\texttt{--enable-shell} option. \textbf{Caution:} You should examine
all input files carefully when using this configuration, other than
plain text: it enables arbitrary code to be run on your computer, both
when running the programme and when compiling the newspaper.}

\section{Glossery of Terms}

Some of these terms are common in newsprint typesetting, others are
specific to this algorithm. An understanding of how these terms will
make the manual easier to follow.

\vfil


\deft{Article}{A collection of prose forming a cohesive story, which
  can be typeset in a number of ways.}

\deft{Article Fragment}{For a paste-up article, this is an element
  consisting of a cut-out section of the article placed on the page.}

\deft{Alley}{Whitespace - or anything else - placed between columns or
  elements on the page.}

\deft{Element}{The term is used here for an article fragment, title,
  or other insert set onto the page.}

\deft{Footers}{Any information provided at the end of an
  article, such as a \textit{``continued on page\dots''} indicatior,
  or footnotes.}

\deft{Headers}{Any information provided at the start of an article or fragment,
  such as headlines, sub-headings, authors and so on. If an article
  breaks across pages, the \textit{``\dots continued from page''}
    indicator is also considered to be a header.}

\deft{Headline}{A title of an article, set in a larger, bold font. The
  reader can scan the page for headlines to determine which articles
  they want to read.}

\deft{Head-span Article}{An article, set onto the page in a rectangle,
  with a headline spanning the entire article.}

\deft{Input}{A general term for an article, input or other file read
  by this programme to produce the output.}

\deft{Input directory}{A filesystem directory (or folder) containing
  one or more inputs. There may be one or more input directories.}

\deft{Insert}{An element added to the page with fixed
  dimensions. Typically, this is a graphic image or title, but the
  algorithm considers anything with a fixed-size to be an insert,
  including a head-span article where the number of columns is known.}

\deft{Newsprint}{The ink used to print the newspaper. The term is used
  here for anything typeset in the style of a newspaper, especially a
  portion of a newspaper page.}

\deft{Output directory}{A filesystem directory (or folder) into which
  the typeset version of the newspaper is placed.}

\deft{Overflow}{When typesetting a page, any elements (especially
  article fragments) that do not fit on a page are known as the
  overflow. Overflow is always typeset at a later place than it would
  normally be expected to.}

\deft{Paste-up}{A method of creating newsprint by setting an entire
  article in a single, long column, then splitting that column across
  the page. Distinguished by having the headline inside the column.
  Originally, this was done by literally cutting and pasting
  to create a master proof.}

%% \deft{Truck}{The detachable portion of a printing press containing the
%%   typeset page. Notably, the ``truck'' is the size of a page, so a
%%   ``double-truck page'' would be a centerfold page twice the size of a
%%   single page.}

\section{Filesystem Layout}

This is an implementation of an algorithm to take \textit{inputs} from
files supplied by the user, and to produce an output layout. The
implementation outputs a \LaTeXe\ source file that references back to
the supplied inputs, and can then be typeset directly to produce a
rendered document. By default, the reference programme will also perform
this typesetting.

The user is responsible for supplying the files required as
inputs. The user must create one or more \textit{input directories} on
the filesystem, which will contain the content in roughly the order of
typesetting. The user must also create an \textit{output directory},
which should have the same parent directory as the first input
direcotry.

This arrangement allows a lot of flexibility. For example: different
sub-editors could supply different directories of inputs for
typesetting, which can then be chained together; or you could supply
one directory for each page of the newspaper, satisfied that any overflow
from one page will still move to the next page as required.

The output from the programme must also go somewhere. It's not
convenient to write the output to the same directory as the input,
because of the way that \LaTeX\ uses multiple temporary files during
its layout --- at minimum, an \verb!.aux! and a \verb!.log! file ---
which may affect subsequent runs, and can clutter the input
directory. So, to make sure we can always get back to a ``clean
slate'' to recreate the newspaper again, we always place the output in
a directory of its own. The user can choose the name of this
directory, but if they don't, a new directory named simply \verb!out!,
will be created.


\section{Settings File}

The settings file is named \verb!settings.properties! and is formatted
as a Java language Properties file, described in full in the Java
SDK\footnote{\texttt{https://docs.oracle.com/en/java/javase/11/docs/api/java.base/java/util/Properties.html\#
    load(java.io.Reader)}}.

Four our purposes, each file contains a number of lines. Blank lines,
and lines starting with a hash/pound sign~(\verb!#!) are
ignored. Settings are given one per line, with the setting name
followed by an equals sign~(=) or colon~(:), followed by the setting
value. The backslash~(\textbackslash) escapes the following character,
so a single backslash in a property value will be encoded as two
backslashes in the property file.

When the application is run, each source directory is checked for a
settings file. It is strongly recommended
that the filename be in all lowercase, for portability. Settings files
must be plaed directly in the source directory; subdirectories are not
checked. If the same setting key is found in more than one settings
file, the value from the later file is used, based on the order in
which the source directories are given to the programme.

Each setting has a default value, which may sometimes be calculated based on
other settings (in which case, changing the other setting will change
the default value).

Where a setting is a length, it should be specified as a
floating-point number, followed directly by a unit indicator. 0
lengths do not require a unit. Units are not case-sensitive.
Supported units are:
\begin{center}
  \begin{tabular}{r|l}
    \verb!in! & inches (1\textrm{in} = $\frac{1}{36}$ International Yard) \\
    \verb!inch! & inches \\
    \verb!inches! & inches \\
    \verb!pt! & \TeX\ points ($1\textrm{pt} = 0.35145980mm$) \\
    \verb!mm! & millimeter \\
    \verb!cm! & centimeter \\
    \verb!dm! & decimeter \\
    \verb!m! & meter
  \end{tabular}
\end{center}

Note that the typesetting units \verb!em! and \verb!ex! are not
currently supported, and calculations cannot be supplied.

\subsubsection*{A note on paper sizes}

In essence, a newspaper is a method of providing a large amount of
accessible text onto a single page. For this reason, newspapers have often
been printed onto larger size paper sheets.

Choosing a paper stock depends on your use-case, the size of paper you can
obtain, and what your printing equipment can support. You may
configure any paper size within the limits of \LaTeX\ (up to
2267.05408in for both \texttt{pageWidth} and \texttt{pageHeight}), but
if you intend to print it, you will need to ensure you have the
available paper and suitable printing equipment. There must be
sufficient space for at least one column and the page must be larger
than your largest fixed-size insert  --- including margins.

In most of the world, home printing uses paper sizes of the finished
``A'' series defined by ISO~216 (and incorporated by ANSI into ASME
Y14.1M). The ``B'' series of the same standard supplies larger paper
commonly used in commercial printing; it is available in larger sizes
and is mainly used for printing ``A''-series paper with bleed margins.

In North America, the ANSI standards are formalised in ASME~Y14.1; the
ARCH system is not a standard, but provides useful paper sizes
preferred by architects.

The following table provides various dimensions of paper from around
the world:-\nopagebreak\vspace*{1.5em}\nopagebreak

\centerline{
  \begin{tabular}{r|c|c|c}
    \multicolumn{4}{c}{\textbf{Common Paper Sizes}} \\
   & \textbf{Code} & \textbf{\texttt{pageWidth}} &
  \textbf{\texttt{pageHeight}} \\
  \hline
  ISO~216 & A3 & 297mm & 420mm \\
  ISO~216 & A4 & 297mm & 210mm \\
  ISO~216 & B2 & 500mm & 707mm \\
  ISO~216 & B3 & 353mm & 500mm \\
  ANSI & Letter & 11in & 8.5in \\
  ANSI & Legal & 14in & 8.5in \\
  ARCH & A & 9in & 12in \\
  ARCH & B & 12in & 18in \\
  ARCH & C & 18in & 24in \\
  ARCH & D & 24in & 36in \\
  ARCH & E & 36in & 48in \\
  ARCH & E1 & 30in & 42in \\
  ARCH & E3 & 27in & 39in
  \end{tabular}
  \hfil
  \begin{tabular}{r|c|c}
    \multicolumn{3}{c}{\textbf{Generic Newspaper Sizes}} \\
    \textbf{Code} & \textbf{\texttt{pageWidth}} &
  \textbf{\texttt{pageHeight}} \\
  \hline
  \textbf{Tabloid}\hspace*{0.25in} & & \\
  CanadaT & 260mm & 368mm \\
  NorwayT & 280mm & 400mm \\
  BritainT & 280mm & 430mm \\
  \textbf{Berliner}\hspace*{0.25in} & & \\
  Midi & 315mm & 470mm \\
  \textbf{Broadsheet}\hspace*{0.25in} & & \\
  USA & 381mm & 578mm \\
  Norwegian & 400mm & 570mm \\
  South Africa & 410mm & 578mm \\
  (A2) Australia/ & \multirow{2}{*}{420mm} & \multirow{2}{*}{594mm} \\
  New Zealand\hspace{1em} & & \\
  Britain &  375mm & 597mm \\
  \end{tabular}
}

\subsubsection*{A note on Fonts and Font Selection}

Choosing fonts is always difficult, particularly where a high level of
readability is required. Newspapers should be quick to skim and easy
to read in detail, in spite of often very small text.

Generally, most documents use only a single typeface, in several
variants (like bold text for titles). \LaTeX\ has notoriously fiddly
font handling, because it's designed around this use-case: it's
possible to change fonts arbitrarily throughout a document, but it's
far more common to simply switch to a variant --- like using
\textit{italic} for emphasis --- temporarily.

For a newspaper, there are generally three distinct typefaces in use:
the main title is typically a bold, gothic style; headlines use a
large serif font that's easily picked out on the page; and the article
text uses a small, narrow-spaced serif font that guides the eye along
the page. (There have been few published studies on the effectiveness
of serif fonts on printed newspapers; however, when viewed on a
backlit electronic display, sans-serif fonts are more readable:
consider your audience).

A font is a physical container of engraved metal letters used in a
printing press; it contains the complete alphabet, including ligatures
and punctuation, for a consistent size, scale and form of the text. Electronic
printing has allowed many thousands of fonts to be represented, and
the term ``font'' has come to mean the specific shapes of the symbols
once held in a single physical font. Notably, \textit{10pt~Times} and
\textit{11pt~Times} are different \textit{fonts}, even though they are
the same \textit{typeface}~(Times).

In \LaTeX, each font has a number of properties:-\par
\deft{Encoding}{This controls how \TeX's internal codes map to each glyph in the font.}
\deft{Family}{The typeface, or group of related typefaces}
\deft{Series}{Weight and width of the font}
\deft{Shape}{Upright, italic, slanted (oblique), small caps, outline, \dots}
\deft{Size}{The point size of the font. As a rule of thumb, the letter
  ``M'' will be this tall when measured in typesetting points --- but
  the exact size is set by the font author.}
\deft{Spacing}{\LaTeX\ actually considers this part of the font size;
  it tells \TeX\ how far apart to set the baselines between each line
  of text within a paragraph. By default, this program leaves \textit{2pt}
  between each line, but this can be varied when selecting the font
  depending on visual cues like the length and frequency of
  descenders~(parts of text extending below the baseline, like in the
  letters ``p'' and ``q''). The aim should be to balance the
  whitespace around the text and a clear distinction between lines,
  against the constant pressure to increase the volume of text on the
  page.}

To use a font with the \textit{\LaTeXe\ Font Selection
  System}~(\texttt{FSS}), it must first be installed on your system;
this typically happens when \LaTeX\ is installed, but there is a
(somewhat non-trivial) process to install your own. This process produces
a file with a ``.fd'' extension for each combination of encoding and
family, and only by inspection of this file can you determine the
available values of series and shape.

%\LaTeX\ also has support for the more common \textit{TrueType
%  Font}~(TTF) system.
If using \textit{XeLaTeX}, or \textit{LuaLaTeX}
you can simply use any font installed on your operating system, by
setting the forllowing options in a settings.properties file. See \textit{The Fontspec
  package}\footnote{\texttt{https://ctan.org/pkg/fontspec}} for
further details:-

\nopagebreak
\begin{verbatim}
preambleFont0=\usepackage{fontspec}
preambleFont1=\setmainfont{Full name of font}
\end{verbatim}

If not using XeLaTeX or LuaLaTeX, the default encoding~(\verb!TU!) may
not be available. For most english-language text, you are recommended
to choose the \verb!T1! encoding, which is the orginal ``standard''
\TeX\ encoding, and widely supported. The \LaTeX\ project has a full
description of the various font
encodings\footnote{\texttt{https://www.latex-project.org/help/documentation/encguide.pdf}}:
the trick is to find encodings that are supported both by the font,
and also by the \LaTeX\ platform you are using: otherwise \TeX\ will
fail to find the characters and replace them with, typically,
\textit{Computer Modern Roman}. To switch the default encoding to
\verb!T1!, add this to your \verb!settings.properties! file:-

\nopagebreak
\begin{verbatim}
defaultFontEncoding=T1
\end{verbatim}

Having changed the main font family to Times, a new problem emerges:
there is no suitably narrow monospaced font in the Times family. Yet
this document supports Markdown, and monospaced fonts are typically
seen in many Markdown documents found online, especially when
discussing computers and data files. The default monospaced font
(\textit{Computer Modern Teletype}) has too wide a spacing for use in
narrow columns, while also being visually out of kilter with
\textit{Times Roman}. It is beyond the scope of this project to
develop a new \textit{Times Teletype} font, so instead the new default
is \textit{Latin Modern Mono Light Condensed} (\textit{Latin Modern
  Mono} is the family; \textit{Light Condensed} is the weight). This
is chosen because in it a universally available high-quality font,
ideally suited for narrow columns, and because no font that is more
visually suited to \textit{Times Roman} currently exists. You can
switch the default teletype font using the following (TODO).


Having carefully selected your fonts, you should create a test paper,
perhaps using the demo input files, and then check the \LaTeX\ output
log file carefully for warnings: it's possible for fonts to be missing
expected glyphs, or even entire shapes or series.

One further warning: The \textit{Mode=Plain} input expects the font
you are using to provide characters for backslash~(\textbackslash),
and braces~(\{\}) at the standard unicode character points. Not all
fonts installed in \TeX\ do this, and so some choices may result in
unexpected characters in the output; if you change the default
document font to one which does not support these characters ---
inclruding \texttt{cmr}, the default font for \TeX --- then these
characters may be omitted in the output, or different characters
displayed instead, when they appear in a \textit{Mode=Plain} input.


Further reading on fonts is available at the \LaTeX\ Font
Guide\footnote{\texttt{https://www.latex-project.org/help/documentation/fntguide.pdf}},
and documentation on many pre-installed fonts is found at the \LaTeX\ Font
Catalogue\footnote{\texttt{https://tug.org/FontCatalogue/allfonts.html}}.




\begin{multicols}{2}

\setting{version}{symantic version number\footnote{\texttt{https://semver.org/}}}{1.0.0}{
  This setting will be used by future releases of this
  programme. Where there are changes that break the visual output of
  the newspaper, setting this version number in your
  \texttt{settings.properties} to the release number of the programme,
  will make a best-effert to produce the
  output of the previous version.\par
  This does not apply where a bugfix corrects clearly-erroneous output
  in a previous release; you are recommended to keep your PDF for
  posterity rathar than regenerating the \texttt{.tex} file each time
  an old PDF is required.
}
\setting{allowTexFileOverwrite}{``true'' or ``false'', any case}{false}{
  Normally, the programme will abort if the output file --- typically
  \texttt{newspaper.tex} --- already exists, to ensure that you can't
  accidentally overwrite the last edition's newspaper. Set this to
  ``true'' if you are going to run the programme repeatedly.\par Always take
  care to back up your final edition newspapers regardless of this setting.
}
\setting{pageSize}{code in above table}{650mm $\times$ 750mm}{
  The \texttt{pageSize} simply sets both \texttt{pageWidth} and
  \texttt{pageHeight} based on the above table. For Australia/New
  Zealand, use code~``A2''.\par This setting changes the default width
  and height of the page, and is overridden by \texttt{pageWidth}
  and/or \texttt{pageHeight}. Note that this is the size of a
  \textit{page}; the \textit{paper} size should normally be twice
  this, to allow four pages per leaf. \par Note that \textit{A4} pages
  are printed in landscape by default, as headers can appear cramped
  on paper much narrower than around 300mm.
}
\setting{pageWidth}{length}{650mm, or from \texttt{pageSize}}{The width of a page, including all
  margins.\par The default page size is a for typical broadsheet
  newspaper; see \texttt{pageSize}.}
\setting{pageHeight}{length}{750mm, or from \texttt{pageSize}}{The height of a page, including all margins.}
\setting{columnWidth}{length}{1.5in}{The width of a newsprint column,
  not including any alleys set to its sides.}
\setting{columnHeightRatioOfPage}{number with decimal places,
  dimensionless units}{0.9}{The ratio of the height of the page to the
  height of a column. Ignored if the \texttt{columnHeight} is
  set. Must be greater than $0$ and no greater than $1$. Sets the top
  and bottom margins of the page.}  \setting{columnHeight}{length}{The
  height of a column. If not specified, it is calculated based on
  \texttt{pageHeight} and \texttt{columnHeightRatioOfPage}}
\setting{alleyWidth}{length}{0.125in}{Space between columns, including
  any dividing lines.}  \setting{alleyHeight}{0.125in}{length}{When
  elements are set above each other, this is the amount of space to
  leave between them, including any dividing lines.}
\setting{alleyThickWidth}{0.0125in}{length}{The width of a vertical
  line to place to the sides of newspaper columns. Set to 0 to disable
  vertical dividing lines.}
\setting{alleyThickHeight}{length}{0.0125in}{The height of a
  horizontal line to place between elements that are placed above each
  other. Set to 0 to disable horizontal dividing lines.}
\setting{minSideMargins}{length}{0.125in}{The minimum amount of
  whitespace to set at the sides of each page. Even if this is set to
  $0$, there will almost always some space at the sides, because the
  alrogithm will only fit entire columns onto the page, with
  fixed-size alleys between them, and may put fewer columns onto a
  page than will fit in order to reduce the chance of orphan columns
  on the last page.\par This is the guaranteed page border, totalled
  over left and right margins.}
\setting{maxSquashVSpace}{length}{0.4in, or 0 for version$\le$0.0.1}{During
  layout, if the amount of space after an article fragment is greater
  than this amount, the fragment will be stretched by this amount, to
  avoid the blank space.\par
  This ensures that the bottom of columns line up, even when the
  natural height of each article fragment differs for any reason.\par
  You should include \textit{rubber space} in the line or paragraph
  skips if this setting is non-zero, to avoid an \textit{underfull
    \textbackslash vbox} warning.
}
\setting{defaultFontSize}{\TeX\ points, typically 10, 11 or 12}{10}{This 
  setting controls the point size of the default document, in
  points.\par
  The \TeX\ distribution provides \textit{class options} (primarily
  spacing defaults) for 10, 11, and 12 points. Where these values are
  not used, the nearest file will be loaded, and some sizes adjusted
  by scaling. For these other sizes, you may need to provide a
  \texttt{defaultFontSizeClo} file to obtain visually perfect output.
}
\setting{defaultFontSizeClo}{String}{\textit{no value}}{This advanced setting
  allows you to change the \textit{class options} for the current
  document, and it will be loaded by the \texttt{fontsize}
  package. You should supply a file named
  \textit{nameSIZE}\texttt{.clo} --- where \textit{SIZE} is the value
  of the \texttt{defaultFontSize} property, and \textit{name} is the
  value of this property. The \textit{.clo} file will then be used to
  set up the various size options for the default font.\par
  For more information, see the FontSize documentation\footnote{\texttt{https://mirror.ox.ac.uk/sites/ctan.org/macros/latex/contrib/fontsize/fontsize.pdf}}
}
\setting{defaultFontEncoding}{\LaTeX\ font encoding name}{TU}{This is
  the font encoding name used when no specific encoding is
  listed. Common choices are \texttt{T1} --- for good compatibility,
  especially in English, when using pdflatex --- \texttt{TU} for
  Unicode fonts supported by Lua\LaTeX and Xe\LaTeX ---
  and \texttt{TS1}, the ``Cork'' encoding, for most fonts originating from the
  \texttt{Metafont} project. \texttt{CJK} covers Chinese, Japanese, and Korean
  characters, while \texttt{X2} covers Cryillic, and \texttt{LGR}
  covers Greek.
}
\setting{defaultFontFamily}{\LaTeXe\ font family code}{ptm}{This is the code
  for an installed typeface on your \LaTeX\ installation. The default is
  \texttt{ptm}, a Times Roman clone. Change this with care, as
  Times is ubiquitous for newspapers.\par Common values are
  \texttt{cmr} for \textit{Computer Modern Roman}, \texttt{cmrs} for
  \textit{Computer Modern Roman Sans}, or \texttt{phv} for a Helvetica
  clone. This can also be a \LaTeX\ macro defined in the preamble that
  expands to the correct name.
  \par Ignored if \texttt{defaultFontFamilyFromHeaders} is changed to
  \texttt{true}.
}
\setting{defaultFontFamilyFromHeaders}{\textit{true} or \textit{false}, case-insensitive}{false}{
  When false, this has no effect.\par
  When true, this disables the attempt to change the value of the default
  typeface after loading the preamble, disabling the
  \texttt{defaultFontFamily}, \texttt{defaultFontSeries},
  \texttt{defaultTeletypeFamily} and \texttt{defaultTeletypeSeries}
  settings. This allows you to import a
  package in the preamble section that changes the default
  typefaces. Otherwise, the typeface will be set to the \TeX\ default,
  \textit{Computer Modern Roman}, which is not well-suited to narrow
  columns.
}
\setting{defaultFontSeries}{\LaTeX\ font series code}{m}{
  This is the series code for the default font. It produces a line in
  the output document \texttt{\textbackslash
    DeclareFontSeriesDefault\[rm\]\{md\}\{m\}} --- where the final
  value is the font series to be used by the default, medium-density
  roman font used in the document.
}
\setting{defaultTeletypeFamily}{\LaTeX\ font family name}{lmtt}{
  This is the default family code for the default
  \textit{teletype}~(monospaced) font. While teletype fonts are rarely
  used in newspapers except for special effects, no good matching
  font currently exists, so this uses the \textit{Latin Modern
    Teletype} font, which supports many languages and the
  ultra-condensed spacing.
}
\setting{defaultTeletypeSeries}{\LaTeX\ font series code}{lc}{
  This setting selects the series (variant) of the teletype
  (monospaced) font. the \textit{Light condensed} variant ---
  supported by \textit{Latin Modern Teletype} --- is used by default,
  as this font (designed for margin notes) works well in narrow
  columns. It is, however, a lightweight font and stands out against
  the Times Roman text; this may or may not be desirable depending on
  usage.
}
\setting{inputFilter}{Comma-separated strings}{\texttt{.tex, .md, .markdown, .txt, .text}}{
  This setting simply defines which files in the input will be checked
  for a \textit{Type} header. This has no bearing on the type of
  content in that file; for that, use the \textit{Mode} header.\par
  Any file that matches one of these values will be considered for
  inclusion; so by default, you can write inputs as: \texttt{.tex};
  \texttt{.md} or \texttt{.markdown}; and \texttt{.txt} or {.text}. It
  is suggested these be used with \TeX, markdown, or plain text, respectively.\par
  It is not possible to include a comma in the filename extension.\par
  There is no wildcard character, whitespace before or after a comma
  is ignored, and blank extensions are ignored.
}
\setting{head\dots}{Font definition}{\textit{18pt bold}}{
  These settings set the font used to typeset the article headlines. They can
  be overridden for each article by changing the corresponding
  \texttt{Font\dots} headers. Note that if \texttt{headCommand} is
  used, it can only be overridden by the \texttt{FontCommand} header;
  otherwise, each header overrides the corresponding setting.
  \par
  The font may be configured using the following options:-\par
  \deft{headEncoding}{The \LaTeX\ encoding code for the text (up to 3 upper-case
    letters, followed by optional digits)}
  \deft{headFamily}{The \LaTeX\ internal name of the font family (typeface). This
    can be a command, such as \texttt{\textbackslash rmdefault} for
    the document default font family, or an internal name like
    \texttt{cmr} for Computer Modern Roman.}
  \deft{headSeries}{The font series code for the font (usually ``m''
    for medium or ``b'' for bold.}
  \deft{headShape}{The shape code for the font (usually ``n'' for
    upright.}
  \deft{headSize}{The size of the font to use, in \LaTeX\ points.}
  \deft{headSpacing}{The spacing of lines in this font; by default,
    $\textit{headSize} + 2\textrm{pt}$. This may be a \LaTeX\ rubber
    length, but only makes a difference if multiple lines of text are
    included.}
  \deft{headCommand}{Some packages provide convenient user commands to
    switch fonts. This header, if specified, will override the other
    \textit{head\dots} headers, and use this command instead.
  }
}
\setting{columnStrategy}{\texttt{balance} or \texttt{fillFirst}}{\texttt{balance}}{
  If set to \texttt{fillFirst}, then each page will be filled with columns before the next
  page is begun, even if there are insufficient columns to fill the paper.\par
  if set to \textt{balance}, then missing columns will be distributed between pages, to ensure
  that the paper looks as balanced as possible.
}
\setting{markdown}{\LaTeX\ header code}{\texttt{\textbackslash
    usepackage\[smartEllipses,fancyLists\]\{markdown\}}}{
    The programme supports accepting input in markdown format using the
    header \texttt{\%\#Mode=markdown}. When such a header is found,
    this setting is automatically added as a line in the document
    preamble.\par
    To successfully compile a markdown article, you will need to set either
    \texttt{latex=lualatex} or \texttt{latex=xelatex}.
    Otherwise, you may need to add
    \texttt{--shell-escape} to the \texttt{latexCmdLine} setting;
    without a version of \TeX\ built on a \textit{Lua} engine, the
    markdown package needs to call a
    supporting programme from \LaTeX. \textbf{Caution: This could be a
      security risk if mixing \texttt{markdown} and \texttt{LaTeX}
      article types in the same document!} it's better to use
    \texttt{lualatex} if in doubt.\par
    The complete list of options are listed in the Markdown package
    documentation\footnote{\texttt{https://mirror.apps.cam.ac.uk/pub/tex-archive/macros/generic/markdown/markdown.html\#options}}. A
    quick reference of the more useful options follows:-\par
    \deft{blankBeforeBlockquote}{Chevrons~($>$) not following a
      blank line will not start a blockquote}
    \deft{blankBeforeHeading}{A heading must follow a blank line.}
    \deft{blankBeforeList}{A list must follow a blank line.}
    \deft{breakableBlockquotes=false}{Ignore blank lines in the middle
      of a blockquote}
    \deft{definitionLists}{After declaring a term, start a new
      line with a colon~(:) to give its definition.}
    \deft{fancyLists}{Change the presentation of lists.}
    \deft{fencedCode}{Enable fenced-code blocks. An open fence is a line
      containing three backticks, and an optional format name, and
      preceeds a block of code. A closing fence line contains three
      backticks, and follows the block of code. The code will be displayed
      as is, perhaps highlighted depending on the format name.
    }
    \deft{hashEnumerators}{Use a hash sign (in US english, a pound
      sign) followed by a fullstop~(\texttt{\#.}) to introduce each item
      of a numbered list.}
    \deft{html}{Allow inline static HTML within the markdown article.}
    \deft{hybrid}{Allow inline \LaTeX\ within the markdown
      article. This can be useful for special effects, but changes the
      meaning of backslash~(\textbackslash) and braces~(\{\}) and
      introduces the possibility of breaking the newspaper compilation.}
    \deft{linkAttributes}{A syntax for including inline images,
      like this: \texttt{![image](foo.jpg)\{\#id .class width=30
        height=20px\}}}
    \deft{lineblocks}{Successive lines starting with a bar~(|) will be
      considered as a block, and leading spaces preserved. Useful for
      addresses.}
    \deft{mark}{Use two equals signs either side of text to
      highlight it by changing the background colour.}
    \deft{notes}{Allows a syntax similar to footnotes, but
      presented inline in the text.}
    \deft{shiftHeadings=\textit{number}}{The given \textit{number} is
      added to the heading level; positive numbers make headings
      smaller.}
    \deft{smartEllipses}{With this option, any occurance of
      \texttt{...} in the input will be properly formatted as an
      ellipsis character~(\dots); without it, you will see three dots
      with poor spacing~(...).}
    \deft{strikeThrough}{With this option, text surrounded with
      double tildes~(\textasciitilde) will be overtyped with a horizontal line.}
    \deft{subscripts}{Text surrounded with single tildes~(\textasciitilde)
      will be shown subscript; e.g. \texttt{H\textasciitilde2\textasciitilde O}
      becomes~H\textsubscript{2}O.}
    \deft{subscripts}{Text surrounded with single carets~(\textasciicircum)
      will be shown subscript; e.g. \texttt{e=mc\textasciicircum2\textasciicircum}
      becomes~e=mc\textsuperscript{2} --- see also \texttt{texMathDollars}.}
    \deft{texComments}{A percent sign~(\%), and any text on a line
      after it, will be ignored. Note that lines starting with a~\%
      will always be ignored due to how the Markdown input is included.}
    \deft{texMathDollars}{When text is surrounded by dollar
      signs, it will be typeset in
      \LaTeX\ math-mode. e.g. \texttt{\$e=mc\textasciicircum2\$} is typeset as
      $e=mc^2$. Double dollars use block math mode.}
    \deft{tightLists=false}{By default, lists lists written as one
      line per item are rendered with less spacing than lists containing paragraphs of
      text. This option disables this tighter-grouping and adds more
      whitespace.}
    \deft{underscores=false}{By default, underscores declare emphasis
      like asterisks; this option disables this. Recommended with the
      \texttt{hybrid} option.}
}
\setting{continuedOnPageText}{\LaTeX}{\textbackslash hfill\textbackslash textit\{\textbackslash small Continued
    on page \%s\textbackslash dots\}}{
  This \LaTeX\ text is displayed after each article fragment where the
  article does not fit completely on the page. If the string contains
  ``\%s'', then it will be replaced with the page number of the
  continuation page.
}
\setting{continuedFromPageText}{\LaTeX}{\textbackslash textit\{\textbackslash small Continued
  from page \%s\textbackslash dots\}\textbackslash hfill}{
  This \LaTeX\ text is displayed before each article fragment where the
  previous article fragment was displayed on a previous page. If the
  string contains ``\%s'', then it will be replaced with the page
  number of the continuation page.
}
\setting{jobname}{filename without suffix}{newspaper}{Primarily, this
  sets the name of the output files written by the alrogithm. Usually,
  this also sets the name of the \texttt{PDF} and \texttt{.log} files
  produced by \LaTeX, although that can be overwritten by setting the
  \TeX\ variable \texttt{\textbackslash jobname} or passing the
  \texttt{-jobname} argument through the \texttt{latexCmdLine}
  setting.}
\setting{lengthsCache}{filename without suffix}{lengths.cache}{
  The name of a file to create in the output directory, in which we
  store the calculated length of each article fragment. This avoids
  the need to repeatedly shell out to LaTeX to calculate lengths of
  each fragment in case the programme is run again.\par
  This cache file is not used by \LaTeX.
 }
\setting{logFile}{filename or path relative to output directory}{layout.log}{
  When creating the output newspaper, a log file is usually created to
  supply detailed information about processing. Depending on how you
  work, this may be more or less convenient that the information
  supplied by the console. This file can be anazylsed to determine why
  something was not processed as expected.\par By default, this file
  is placed in the output directory and named
  \texttt{layout.log}. This setting simply changes the filename.\par
  See \texttt{logFileLevel} for how to configure the contents of this
  file.\par If this file is not created, the console output should be consulted.
}
\setting{logFileLevel}{One of: \texttt{silent}, \texttt{quiet}, \texttt{elements},
  \texttt{algorithm} or \texttt{dump\_all}}{algorithm}{
  This setting determines what is logged to the log file. If
  \texttt{silent}, no log file is created. \texttt{Quiet} will ensure
  the file only contains error messages, \textit{i.e.} the reason the
  output \texttt{.tex} file could not be produced. \texttt{Elements}
  will provide statistics for each inputs file (this could perhaps be used to
  determine how many column-inches of content have been supplied by
  each author). \texttt{Algorithm} will trace the algorithm as it
  considers how to lay out each page, which can be useful for editors
  looking to better understand and tweak the
  way pages are laid out. \texttt{Dump\_all} can produce a large amount of content,
  and includes the text of all intermediate \texttt{.tex} files used
  to measure the length of articles.
  \par
  Each level implies the output of previous levels as well.
  \par
  NB: The \texttt{Algorithm} level includes an estimate of how much
  more content is required to completely fill the number of pages
  needed to set the provided input. This is an over-estimate, as
  it does not allow for the column-inches used by ``continued on'' labels,
  ``continued from'' labels, or alleys between articles.
}
\setting{stdOutLevel}{One of: \texttt{silent}, \texttt{quiet}, \texttt{elements},
  \texttt{algorithm} or \texttt{dump\_all}}{elements}{
  This setting is identical to \texttt{logFileLevel}, but controls the
  output produced on the console --- technically, on the
  \textit{standard output stream}.\par Please note that errors parsing
  the settings file will always appear on the console via the
  \textit{standard error} stream; see \texttt{stdErrLevel}.
}
\setting{stdErrLevel}{One of: \texttt{silent}, \texttt{quiet}, \texttt{elements},
  \texttt{algorithm} or \texttt{dump\_all}}{silent}{
  This advanced setting is of interest to users wishing to automate
  the process. The setting is identical to \texttt{stdOutLevel},
  except that it controls the \textit{standard error stream}, which
  remains displayed on the console even if the standard output is
  redirected.\par
  \texttt{Silent} here means to include only the most critical of
  errors, such as parsing the settings files: such messages will
  always be sent to \textit{standard error}. \par
  Any messages sent to \textit{standard error} will not be sent to
  \textit{standard output}, regardless of the \texttt{stdOutLevel}
  flag. For example, if \texttt{stdErrLevel} is set to
  \texttt{quiet}, and \texttt{stdOutLevel} to \texttt{elements}, then
  all errors will be sent to \textit{standard error} and statistics
  will be sent to \textit{standard output}.
}
\setting{latex}{\textit{OS-specific} path-name}{lualatex}{This is the executable used to
  size articles and lay out the final document. For security, it is
  best to set this to the full path to the \LaTeX\ executable on your
  system. Using a \texttt{pdflatex} or \texttt{lualatex} executable is
  recommended, as the original \texttt{dvi} format output may cause
  issues with some fonts, including the default Times clone.}
\setting{latexCmdLine}{\LaTeX\ command-line options}{\texttt{--interaction=nonstopmode}}{
  This advanced setting changes how the programme invokes the
  \LaTeX\ command, both for the final output, and for the measuring of
  element sizes. See the \LaTeX\ documentation for full details.\par
  For example, you could add \texttt{-jobname=edition1} to change the
  output filenames to \texttt{edition1.tex} and \texttt{edition1.pdf} and
  the \LaTeX\ log filename to \texttt{edition1.log}.
}
\setting{texinputs}{Environment variable}{\texttt{:}}{This setting
  overrides the \texttt{TEXINPUTS} enveronment variable when running
  the \LaTeX\ programme. \LaTeX\ interprets this as a colon-delimited
  list of absolute or relative pathnames to search for inputs.\par
  Adding the absolute path of an input source, then a colon, may solve a number of
  problems when an input source cannot be found.\par
  This is also useful to inform \TeX\ of the location of additional
  packages that you have downloaded but not installed on your system.\par
  Ordinarily, this property should end in a trailing colon, to
  indicate to \LaTeX\ that your system's installation path should be
  used for any standard includes.\par
}
\setting{enableLaTeXHooks}{``true'' or ``false'', any case}{false}{
  \textbf{Caution:} Enabling this setting will allow all documents to
  run arbitrary \LaTeX\ code. This may be a security concern if you do
  not trust your authors.\par
  This option permits the use of the \texttt{BeforeBox}, \texttt{BeforeContent} and
  \texttt{AfterContent} \textit{headers} in articles, which allows the
  injection of arbitrary \LaTeX\ before and after articles (including
  plain-text and markdown). Any output generated by these headers will
  be prepended or appended (respectively) to the final article. The
  \LaTeX macros are not placed in a group.
}
\setting{inputWithoutCopy}{``true'' or ``false'', any case}{false}{
  In most cases, the program will copy the text from the input files
  into the output \texttt{.tex} file. If the input files are each
  stand-alone (\textit{i.e.} use to external graphics or text files
  themselves), this produces a stand-alone document. \par
  Some users may prefer to set \texttt{inputWithoutCopy=true}, to link
  the input files to the output file instead. This may allow a faster
  turnaround for small edits, as the input files can be tweaked and
  the output file run thourgh \LaTeX\ without invoking the programme
  again. (Note, however, that the column-inches for each article would
  not be re-calculated, possibly resulting in articles overflowing or
  underfilling columns).\par
  This mode is currently ignored by \texttt{mode=plain} articles,
  which do not support the copying mode.
}
\setting{tolerance}{dimensionless integer, between 0 and 10000 inclusive}{500}{
  This is an advanced option, and sets the value of the \TeX\ setting
  \texttt{\textbackslash tolerance}. This must be no less than $0$,
  and values over $10000$ will have no effect. Increasing this number
  will allow extra spacing between words to produce a better layout of
  text within a column.}  \setting{out}{out}{pathname}{This is the
  path-name used to resolve the output directory, where the output and
  logs of the programme are placed. The path is resolved on the
  default filesystem and path: for most systems, this means it is
  relative to the working directory.}
\setting{pretolerance}{dimensionless integer, between -1 and 10000 inclusive}{-1}{
  This is an advanced option, and sets the value of the \TeX\ setting
  \texttt{\textbackslash pretolerance}. This should be less than
  \texttt{tolerance} in most cases. If not set to -1, it instructs
  \TeX\ to first attempt to break paragraphs into lines without
  hyphenation; if this produces an adequate result, the full
  line-break alrogithm is bypassed. This has the effect that fewer
  hyphenated line-breaks will appear in the final document, without
  much loss in quality.\par
  Generally, this setting is not useful in newspapers, because the
  text is too narrow to typeset effectively without
  hyphenation. However, if you have an especially narrow font, or many
  especially wide columns, then this setting may compile slightly
  faster with some fewer hyphenation points.
}
\setting{emegencystretch}{\LaTeX\ function code}{\texttt{\textbackslash emergencystretch = 0.1 \textbackslash hsize}}{
  This is an advanced option, and controls how the value of the \TeX\ setting
  \texttt{\textbackslash emergencystretch} should be set.\par
  When setting text in narrow columns, there is much less chance of a
  natural line-break position (space or hyphenation point) occurring
  just before the end of a line. \TeX\ responds to this by producing
  overfull lines, with warnings in the output file, and expects the
  user to resolve these warnings by changing the text appropriately so
  that the line-break positions are moved. In full-width text, this
  happens rarely, and the addition of an extra comma or filler word is
  usually all that is needed to resolve the issue, guaranteeing
  high-quality text output. However, this
  becomes very onerous for newspaper text, due to the much-increased
  number of lines, and the lower chance of a good line-break
  combination.\par
  In 1989, to avoid this problem, \TeX\ introduced the
  \texttt{\textbackslash emergencystretch}
  register (in version 3.0). This register tells \TeX how much extra
  whitespace is permissible, per line, before it counts the line as
  being bad and tries another combination of breaks. Higher settings
  mean fewer overfull lines for the user to manually fix, at the cost
  of extra horizontal whitespace.\par
  As a rule of thumb, \texttt{\textbackslash emergencystretch} can be
  larger when there are more average spaces per line in the input.
  This setting holds the definition of a \LaTeX\ function, taking as
  parameters: the number of columns~(\#1), and the current width of
  the text on a column~(\textit{\textbackslash hsize} or \#2). It
  should produce no output when expanded, but set the value of the
  \textit{\textbackslash emergencystretch} register instead. This
  syntax is compatible with the \texttt{multicols} environment, and
  means that an appopriate setting can be used even when the column
  width changes (see the \texttt{innerCols} setting).\par
  An comprehensive article on choosing an optimal value is given in the
  \TeX\ Users' Group magazine, \textit{TUGboat}, volume 38, from p.65\footnote{
  \texttt{https://tug.org/TUGboat/tb38-1/tb118wermuth.pdf}} \par
  It should be noted that the \textit{multicols} environment ---
  principly used for 2--3 columns on A4 or letter paper, perhaps more
  in landscape mode --- uses the value of \textit{4pt per column}, and
  tends to produce very visually spaced output. The authors warn that
  this may not be optimal, and it seems to have been intruduced when
  the number of columns was limited to 10, so may not even have been
  tested with dozens of columns like on a newspaper.
  \par
  The current default of \texttt{0.1\textbackslash hsize} --- one
  tenth of the column with, excluding alleys --- seems to
  work well with even large numbers of columns.
}
\end{multicols}

\setting{preamble\dots}{\LaTeX\ preamble code}{\textit{(see description)}}{
  Each setting beginning with ``preamble'' is added to the document
  preamble.\par
  Note that the settings are sorted lexicographically, so
  \texttt{preamble10} will run before \texttt{preamble9}, because
  \texttt{1} comes before \texttt{9} in the dictionary --- regardless of the order of
  lines in the \texttt{settings.properties} files.\par
  It is also possible to add preamble lines in the individual header
  files.\par
  While writing a \LaTeX\ preamble is relatively advanced, it provides
  almost complete control over your document. A number of defaults are
  provided, which should not normally be changed. There are:-\par
  \begin{quote}
    \texttt{preamble!00head=\textbackslash usepackage\{indentfirst\}}
  \end{quote}
  to indent the first line of each paragraph;\par
  \begin{quote}
    \texttt{preamble!01head=\textbackslash usepackage[british]\{babel\}}
  \end{quote}
  to allow hyphenation within words;
  \begin{quote}
    \texttt{preamble!02head=\textbackslash usepackage[utf8]\{inputenc\}}
  \end{quote}
  to switch the input encoding to \texttt{UTF-8};\par
  \begin{quote}
    \texttt{preamble!03head=\textbackslash usepackage\{newtxmath,nextxtext\}}
  \end{quote}
  to switch the default font to use the new Times package (which must
  come after \texttt{babel}; and
  \begin{quote}
    \texttt{preamble!04head=\textbackslash usepackage\{csquotes\}}
  \end{quote}
  , which does nothing by default, but allows you to add
  ``\textbackslash MakeInnerQuote\{\"\}''to allow the user to type a
  single " character in most places where a quotation mark would
  appear (change ``MakeInnerQuote'' to ``MakeOuterQuote'' for
  American-style quotation marks, where single-quotes are used around
  quotations). At time of writing, there seem to be compatibility
  issues with head-span articles in markdown when using \textbackslash
  MakeInnerQuote.\par
  The use of underscores~(\texttt{!}) in the propery keys indicates
  that these are default settings; any new defaults in future versions
  are likely to be named \texttt{preamble!04head} or similar, so while
  you may use these names to override the default settings, you are
  recommended to choose your own distinct convention for any
  additional lines, to avoid
  conflicts in future.
}

\section*{Inputs}

The programme takes inputs corresponding to the logical inputs
accepted by an editor: newspaper articles, illustrations, title lines
and so on. Each of these inputs corresponds to a file in the input
directory, using a defined extension (\textit{i.e.} the end of the
filename), with a special header to configure
how the document is to be laid out.

Each input produces one or more \textit{elements} --- which may be
\textit{article fragments} or fixed-size \textit{inserts} --- that the
algorithm will place on the page.

Each file contains the metadata to configure how it's to be treated by
the layout algorithm in a number of \textit{header lines}. Header
lines change the behaviour of the layout programme in producing the
typeset newspaper, but not the behaviour of how \TeX\ typesets the
final document. So each header line will begin with \verb!%#! as
\TeX\ will ignore any input line starting with a \verb!%! character,
so any line starting with a \verb!%! but not \verb!%#! can still be
used for comments and will be ignored.

Only the first 50 lines of the document may contain header lines;
header lines after this will be ignored.

Headers consist of a number of key-value pairs. The form is to use the
Java \verb|.properties| standard, but with the addition of \verb!%#!
at the start of every line. Thus, headers form a list of key-value
pairs, similar to the \verb!settings.properties! input file(s).

For example, the following complete input file produces a \textit{paste-up} article
consisting of nonsense pseudolatin:-

\begin{quote}
\begin{verbatim}
%# Type: article
%# Head: Lorum Ipsum
%# Preamble: \usepackage{lipsum}
\lipsum
\end{verbatim}
\end{quote}

The allowable header values are:-\nopagebreak

\begin{multicols}{2}
\header{Type}{case-insensitive}{\textit{None}}{
  One of the following:-\par
  \deft{Article}{A paste-up article}
  \deft{Fixed}{A fixed-size insert. \textit{Width} and \textit{Height} are required.}
  \deft{HeadSpan}{An article. \textit{Cols} is required. The headline will
    span the number of columns given by \textit{Cols}, and the length
    of the article will be as long as required to balance the content
    among this number of columns. The article will be set as a unit
    and not broken up.\par This currently uses the
    \texttt{multicol} package, which is a part of the standard
    \LaTeX\ distribution.}
  \deft{Title}{Intended for the headline at the start of the paper;
    this is a full-width insert displayed in headline font, that can
    be customized through various headers.}
  Other input types may be added in future.

  A document that does not declare a \texttt{Type} header will not
  produce an input in the document. This allows support for
  complex inputs consisting of multiple files.
}
\header{Asset\dots}{filename}{\textit{None}}{
  This is an advanced option that may be useful in fixing the issue of
  articles overrunning their allocated space, if you include many
  files together to form your newspaper article.\par
  Normally, each article is copied to a temporary directory and
  compiled, with \texttt{\textbackslash
    input@files} set to the input folder(s). In most cases, this is
  sufficient to accurately render the document. However, when using
  third-party packages to include files --- particularly through
  Markdown, but sometimes with other packages as well --- the file may
  be missing, and the article will not have the size it needs in the
  final document.\par
  In this case, you can list your assets in the Header file, with
  keys starting ``Asset'', and values equal to the filenames relative
  to the input folder. The assets will be copied to the temporary
  directory and used to calculate a more accurate article size. Use
  one asset setting per file.
}
\header{Mode}{case-insensitive}{LaTeX}{
  One of the following:-\par
  \deft{LaTeX}{This is the default. The text of the article will be used
    directly in the output \TeX\ file. Characters such as
    \texttt{\textbackslash, \{ and \}} have special meaning, and failure to
    follow \LaTeX\ conventions will break the document --- but any number
    of special effects are possible. See the \texttt{markdown} setting
    for further details.}
  \deft{Plain}{Input is plain \texttt{utf-8} text, with paragraphs separated
    with blank lines. Hyphens~(-), backticks~(\textasciigrave) and apostrophies~(\textasciiacute) will be
    treated the same as in \TeX. The text will be formatted as a regular article, but
    special effects (like \textit{italic text}) are not
    possible. Please see also the warning in the note on Fonts and
    Font Selection.}
  \deft{Markdown}{Input is in a simple \texttt{markdown} encoding.
    This lets your authors use basic decorations, such as adding
    asterisks around text to make it \textbf{*bold*}, or underscores
    for \textit{\textunderscore italic\textunderscore}. Markdown is a
    widely adopted de-facto standard format.}
}
\header{Cols}{Natural number}{\textit{Required} for \textit{Type=Headspan}}{
  With \textit{Type=HeadSpan}, this gives the width of the insert, in
  columns. The article will exactly fill the stated number of
  columns.\par
  This uses the ``multicols'' environment, which has
  a built-in limit of 20 internal columns. You can split the document
  across more than 20 columns of the newspaper, but fewer columns than
  expected may appear within the box.
}
\header{InnerCols}{Natural number, at least 2}{Defaults to \textit{Cols}; used only with \textit{Type=Headspan}}{
  With \textit{Type=HeadSpan}, this gives the number of columns to
  divide the article into to typeset; unlike \textit{Cols} which
  determines the size of the article on the newspaper grid. By
  default, if there are up to 20 columns, the article will match the
  rest of the newspaper grid.\par
  This setting can be used to set this article in columns of a
  different width than the rest of the paper.\par
  If this setting is more than 20 columns, then the \texttt{multicol}
  package will report a warning, and 20 columns will be used. This is
  a limitation of \texttt{multicol}.
}
\header{ColumnHint}{Natural number}{\textit{None}}{
  This option allows some influence over the horizontal postiion of
  \textit{Type=Fixed} and \textit{Type=HeadSpan}. It specifies the
  preferred starting column for the insert. Note that the layout of
  such fixed-size elements on the page does allow for some movement
  after the initial placement, so this is not guaranteed to produce
  the expected result in the output.\par
  The left-most column is column ``1'', and columns count upwards to
  the right.
}
\header{RuleWidth}{Length}{\textit{alleyThickWidth} setting}{
  May be used with \textit{Type=HeadSpan} to change the width of the
  rules between the columns within the head-span article. Set to 0 to
  remove lines within the article itself.
}
\header{Width}{Length}{\textit{Required} for \textit{Insert}}{
  Used with \textit{Type=Insert} to set the width of the fixed-size
  insert. Extra whitespace may be added to the left and right of the
  insert, to allow for partial columns.
}
\header{Height}{Length}{\textit{Required for \textit{Insert}}}{
  Used with \textit{Type=Insert} to set the height of the fixed-size
  insert.
}
\header{Head}{Text}{\textit{No headline}}{
  Not used with \textit{Title} or \textit{Insert} types.
  Sets a headline to be set on top of an \textit{Article} or
  \textit{HeadSpan}. \textit{Article} headlines may be set inside the
  column, while \textit{HeadSpan} will span the article. The headline
  font can be chaged for each article by \texttt{Head\dots}, which
  operates the same as the \texttt{head\dots} setting.
}
\header{BoxRule}{Length}{0.0}{
  This applies to \textit{Type=Fixed} and \textit{Type=HeadSpan}
  inputs. If non-zero, \texttt{\textbackslash fboxwidth} will be set
  to this length, and the article will be surrounded in an
  \texttt{\textbackslash fbox}, creating a simple closed line-border
  of this thickness. This is a simple way to make individual elements
  stand out from the surrounding text. See also \textit{BoxSep}.
}
\header{BoxSep}{Length}{3pt}{
  This applies to \textit{Type=Fixed} and \textit{Type=HeadSpan}
  inputs, but only where \textit{BoxRule} is non-zero. This specifies
  the gap between the page element and the bounding box. The value of
  \texttt{\textbackslash fboxsep} will be set to this length while
  rendering the input.\par Note that the width (and height, for
  \textit{Type=Fixed}) of the article are reduced by \textit{BoxRule}
  and \textit{BoxSep} on both edges.
}
\header{Preamble}{\LaTeX\ header code}{\textit{None}}{
  This applies to all input types, and it itstructs the programme to
  add an output line to the preamble of the final document. This
  can be used to load packages required by the article.\par
  \textit{Caution:}\LaTeX\ provides few ways to stop one article from
  interfering with another, and by its nature, the preamble can affect
  the entire document. Therefore, it is recommended to limit using
  \textit{Preamble} to importing macro packages, and to put anything
  else into the settings file.
}
\header{Page}{Integer}{\textit{None}}{
  Sometimes it's useful to put certain content on certain pages:
  headline news is reserved for the first page; sports and comics go
  to the back. This is an editorial decision affecting layout, and can
  be affected by this option.
  
  When this is omitted, or set to zero, it has no effect: articles
  will be set in the order they appear in the file listings, one
  directory at a time. When set to a non-zero value, it specifies the
  minimum page number on which the article will appear, either from
  the front or end page of the newspaper.
  
  Where the value is a (positive) whole number, it gives the minimum
  ``natural'' page number; i.e. couning the first page
  as ``1'', the second as ``2'', and so on.

  Where this is set to a negative integer, this also specifies the
  minimum page number, but using a different count: the ``natural''
  last page is ``-1'', the penultimate page is ``-2'', and so on,
  counting down and backwards. NB: The number of pages here is based
  on the simple area of articles to be typeset. Extra pages may be
  created if this setting is used to create gaps: \textit{e.g.} if all inputs
  have either ``Page: 2'' or ``Page: -1'', then the first page may be
  blank; also, in some rare cases, the final article may overflow the
  last page, and an additional page will be created \textit{after} the
  articles are placed. This additional pages would be numbered as page
  ``0'', ``1'', and so on, as they cannot be counted until after the
  document is complete.

  There is currently no support for double-truck/''centerfold'' pages;
  however, it is expected that these would be counted as a single
  page.

  Note that this option may leave gaps in the newspaper, and that it
  can only make an article appear later than it would otherwise.
}
\header{Date}{\LaTeX\ text}{\textit{localised date of Java programme run}}{
  Used with \textit{Type=Title} to add an extra line, left-aligned,
  just above the title inself. This might declare the date of
  the newspaper edition.}
\header{Edition}{\LaTeX\ text}{\textit{empty}}{
  Used with \textit{Type=Title} to add an extra line, right-aligned,
  just above the title inself. This might declare the price of
  the newspaper.}
\header{Price}{\LaTeX\ text}{\textit{empty}}{
  Used with \textit{Type=Title} to add an extra line, right-aligned,
  just below the title inself. This might declare the price of
  the newspaper.}
\header{TagLine}{\LaTeX\ text}{\textit{empty}}{
  Used with \textit{Type=Title} to add an extra line, left-aligned,
  just below the title inself. This might declare the motto of
  the newspaper.}
\header{Font\dots}{Font definition}{\textit{54pt Almedera Bold}}{
  Used with \textit{Type=Title} to configure the font for the title
  text itself. See the \texttt{head\dots} setting (in the settings
  section) for a description of the available headers.
}
\header{Head\dots}{Font definition}{\textit{none}}{
  If any of the \texttt{Head\dots} font headers are specified, the
  headline for the article will be set in the supplied font. This
  allows different styles for each headline, in the manner of a
  tabloid. Each overrides the corresponding \texttt{head\dots} setting.
  \par
  If \texttt{headSize} is overridden by \texttt{FontSize}, the default
  value of \texttt{FontSpacing} becomes $\textit{FontSize} +
  2\textrm{pt}$.
  \par
  While modern broadsheets do not typically use different headline
  fonts per article, historical newspapers certainly did: using larger
  headlines to draw the eye to more important stories.
  \par
  It is suggested to consider \textit{Headseries=bx} on wider
  \textit{Type=Headspan} articles, where this font definition is provided.
}
\header{Date\dots}{Font definition}{\textit{8pt Times}}{
  Used with \textit{Type=Title} to configure the font for the date
  line itself. This works identically to \texttt{Font\dots}, but
  replacing the prefix ``Font'' with ``Date''.
}
\header{Price\dots}{Font definition}{\textit{8pt Times}}{
  Used with \textit{Type=Title} to configure the font for the price
  line itself. This works identically to \texttt{Font\dots}, but
  replacing the prefix ``Font'' with ``Price''.
}
\header{Edition\dots}{Font definition}{\textit{8pt Times}}{
  Used with \textit{Type=Title} to configure the font for the edition
  line itself. This works identically to \texttt{Font\dots}, but
  replacing the prefix ``Font'' with ``Edition''.
}
\header{TagLine\dots}{Font definition}{\textit{8pt Times}}{
  Used with \textit{Type=Title} to configure the font for the tag
  line itself. This works identically to \texttt{Font\dots}, but
  replacing the prefix ``Font'' with ``TagLine''.
}
\header{LeftBox\dots}{\LaTeX\ text}{\textit{empty}}{
  Used with \textit{Type=Title} to add the given content centered in a
  box to the left of the title. Such boxes are typically used to
  display logos or advertise pages of potential interest to the
  reader.

  Additionally, the following related headers can configure the box
  itself:-\par
  \deft{LeftBoxWidth}{ Sets the width of the box; default \textit{1in}. }
  \deft{LeftBoxHeight}{ Sets the height of the box; default is
    \textit{LeftBoxWidth} for a square box.}
  \deft{LeftBoxRaise}{ Sets the vertical offset of the box's content. Default is
    given by $\frac{\mbox{\textit{LeftBoxWidth}}}{2} -
    0.05\mbox{in} = 0.45$in, assuming the default LeftBoxWidth.
  }
  \deft{LeftBox\dots}{ You can also set the font for the left box by
    specifying \texttt{LeftBoxCommand} or \texttt{LeftBoxFamily} etc;
    see \texttt{Font\dots}. Note that the default is the document main
    font.}
}
\header{RightBox\dots}{\LaTeX\ text}{\textit{empty}}{
  As \textit{LeftBox\dots} except that this applies to a box shown to
  the right of the title. The lengths for each box must be configured
  separately.
}
\header{RaiseLength}{Length}{\textit{10pt}}{
  The top line of the title (date and edition) is not allocated any
  space, as to do so would make the newspaper title appear too low on
  the page.\par This setting is the distance that the top line should
  be raised up, into the page borders. The value of
  \texttt{HeadSpacing} is typically a good starting point for tweaking
  this value.
}
\header{BeforeBox}{\LaTeX\ text}{\textit{empty}}{
  Unless the setting \texttt{enableLaTeXHooks} is set to true, this
  option does nothing.\par If the setting \texttt{enableLaTeXHooks} is
  set to true, this header causes the \LaTeX\ argument to be prepended
  to the start of the article, before the article's
  \texttt{\textbackslash vbox} or \texttt{\textbackslash fbox} (thus
  appearing above the header, if any). This can only be used with
  \textit{Type=Headspan}.
  }
\header{AfterBox}{\LaTeX\ text}{\textit{empty}}{
  Unless the setting \texttt{enableLaTeXHooks} is set to true, this
  option does nothing.\par If the setting \texttt{enableLaTeXHooks} is
  set to true, this header causes the \LaTeX\ argument to be appended
  to the end of the article, after the article's
  \texttt{\textbackslash vbox} or \texttt{\textbackslash fbox} (thus
  appearing above the header, if any). This can only be used with
  \textit{Type=Headspan}.
  }
\header{BeforeContent}{\LaTeX\ text}{\textit{empty}}{
  Unless the setting \texttt{enableLaTeXHooks} is set to true, this
  option does nothing.\par If the setting \texttt{enableLaTeXHooks} is
  set to true, this header causes the \LaTeX\ argument to be prepended
  to the start of the article. This can be used with any input mode.
}
\header{AfterContent}{\LaTeX\ text}{\textit{empty}}{
  Unless the setting \texttt{enableLaTeXHooks} is set to true, this
  option does nothing.\par If the setting \texttt{enableLaTeXHooks} is
  set to true, this header causes the \LaTeX\ argument to be appended
  to the end of the article. This can be used with any input mode.
}
\header{MagnetNorth}{Percentage}{\textit{empty}}{
  Not to be used with \texttt{Type=article}.

  When specified, an additional step is performed after laying out the
  fixed inserts on the page, and before laying out paste-up articles.

  The total free space directly above and below this insert is
  calculated, using the minimum across columns if there is more than
  one column. The document is then moved vertically so that this
  percentage of this total free space appears above the document.

  Thus, if you say \texttt{MagnetNorth: 100\%}, the insert will be
  ``attracted'' to the top of the page, and move as far as possible in
  that direction.

  Only one pass is made and each article is moved separately, but in
  general this is sufficient to allow many manual tweaks of the layout.
}
\header{MagnetSouth}{Percentage}{\textit{empty}}{ As
  \texttt{MagnetNorth}, but the meaning is reversed --- thus
  \texttt{100\%} means to move the article down as far as possible on
  the page.

  \texttt{MagnetNorth} and texttt{MagnetSouth} may not both appear on the same file.
}
\end{multicols}

\section*{Command-Line}

The programme is distributed as a \verb!.jar! file: an archive of
executable Java code. To run the programme, you will need to execute
the \verb!java.exe! executable:

\begin{quote}
\begin{verbatim}
  java.exe -jar layout.jar srcdir1/ srcdir2/
\end{verbatim}
\end{quote}

Here, \verb!srcdir1! and \verb!srcdir2! are paths to the
\textit{source directories} containing your newspaper inputs.

This will show a console window, showing some information about the
newspaper layout, and any errors that are raised by \LaTeX.

All of this information shown on the console is also available in a
log file inside the output directory, provided that the output
directory setting can be read, and points to a usable directory. You
may prefer to disable the console window; in which case, use the
windowless version~\texttt{java.exe} instead:-

\begin{quote}
\begin{verbatim}
  javaw.exe -jar layout.jar srcdir1/ srcdir2/
\end{verbatim}
\end{quote}

Note that, without a console window, errors relating to discovery of
the output directory or creation of the log file, will not be shown.

\end{document}
